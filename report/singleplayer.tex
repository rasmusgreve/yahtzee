To calculate the optimal solitaire strategy we can exploit that we don't have to calculate all states for each possible combination of categories and score. Instead we only need to keep track of what categories that have been chosen. This equals to finding every bit combination given 13 bits, one for each category. The only exception for this rule is the bonus - we therefore need to keep track of whether the score of the 6 upper categories are $\geq 63$ or higher\footnote{The condition for scoring a bonus}. This can be done using 6 extra bits meaning there are $2^{19} = 524,288$ different states. We should therefore calculate the expected value for each of these states to create the optimal solitaire strategy.
%TODO: DAFUQ IS A WIDGET????
Each state represents the start of a players turn, we therefore have to analyse the whole outcome of one players turn for a current state to find the expected value for a single state. For each state there can be performed a tree search to the next state, we will refer to this search as the \emph{big dynamic search}. To analyse how to play a turn a tree search can be performed to analyse the potential outcomes of the three rolls and two keeps consisting of a turn. We will refer to this search as the \emph{small dynamic search}. It can be proven that this tree search of a state consists in total of 1681 internal states and 21000 internal edges between them\footnote{See the section \emph{Sizing up the graph} in \emph{James Glenn An optimal strategy for Yahtzee}\cite{glenn2006optimal}} and therefore calculating the value of all states consists of a tree search of exploring $2^{19} * 21000 = 11\mbox{billions}$\footnote{If properly optimized this tree search can be performed in about an hour} edges. 

By caching the resulting value for each state the result of the internal tree search made for each state can be reused and therefore only have to be performed once meaning that after the initial calculation of the solitaire strategy the game can be played in seconds.