We can calculate the optimal strategy for maximizing a players score in yahtzee by exploiting that the state space is small enough to analyse completely through tree search. Mainly we can exploit that we don't have to calculate all states for each possible combination of categories and score. Instead we need to only keep track of what categories that have been chosen. This equals to finding every bit combination given 13 bits, one for each category. The only exception for this rule is the bonus - we therefore need to keep track of whether the score of the 6 upper categories are $\geq 63$. This can be done using 6 extra bits meaning there are $2^{19} = 524,288$ different states. We should therefore calculate the expected value for each of these states to create the optimal single player strategy. \newline \hspace*{20px} Each state can be defined as a widget representing the start of a players turn. For each widget there can be performed a tree search to the next widget, that is a tree search of the three rolls and two keeps consisting of a turn. It can be proven that a widget contains in total 1681 states and 21000 edges between them\footnote{See the section \emph{Sizing up the graph} in \emph{James Glenn An optimal strategy for Yahtzee}\cite{glenn2006optimal}} and therefore calculating the value of each widget is $2^{19} * 21000 = 11\mbox{billions}$\footnote{If properly optimized this tree search can be performed in about an hour}. \newline \hspace*{20px} The single player strategy doesn't play yahtzee optimally since it disregards the board of the other player(s). It is considered the best as of yet documented AI for yahtzee and will therefore be used as comparison against our suggestion for a better AI. 