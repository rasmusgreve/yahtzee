To calculate the optimal solitaire strategy we can exploit that we don't have to calculate all states for each possible combination of categories and score. Instead we need to only keep track of what categories that have been chosen. This equals to finding every bit combination given 13 bits, one for each category. The only exception for this rule is the bonus - we therefore need to keep track of whether the score of the 6 upper categories are $\geq 63$. This can be done using 6 extra bits meaning there are $2^{19} = 524,288$ different states. We should therefore calculate the expected value for each of these states to create the optimal solitaire strategy.

Each state can be defined as a widget representing the start of a players turn. For each widget there can be performed a tree search to the next widget, that is a tree search of the three rolls and two keeps consisting of a turn. It can be proven that a widget contains in total 1681 states and 21000 edges between them\footnote{See the section \emph{Sizing up the graph} in \emph{James Glenn An optimal strategy for Yahtzee}\cite{glenn2006optimal}} and therefore calculating the value of each widget consists of a tree search of exploring $2^{19} * 21000 = 11\mbox{billions}$\footnote{If properly optimized this tree search can be performed in about an hour} edges. 

By caching the whole result the calculating can be reused and therefore only have to be performed once meaning that after the initial calculation of the solitaire strategy the game can be played in seconds.