Within the field of game theory there exists games with state spaces so small, that the optimal strategy can be computed within hours - such as simple games like Tic-Tac-Toe or more complex games like \st{Checkers}.
In the other end we have games with state spaces so large that exploring the entire state space is impossible to do within a reasonable amount of time - like Chess or Go.
The focus of this project lies on the game \emph{Yahtzee} that in its solitaire (single-player) version has a state space belonging to the first category, but in multi-player version belongs to the second.

\begin{table}[h] %Den står et lidt underligt sted?
\centering
\begin{tabular}{|>{\columncolor{Gray}}r|r|r|}
\hline
\rowcolor{Gray}
 & \textbf{Solitaire} & \textbf{2 players} 		\\ \hline
\textbf{Bits} 	& 22		&  49 					\\ \hline
\textbf{States} & 4,194,304	&  562,949,953,421,312	\\ \hline	

\end{tabular} 
\caption{The number of bits required to represent a state and the resulting upper bound of the state space size in a \emph{Yahtzee} game.}
\end{table}

An optimal solution for the game of solitaire \emph{Yahtzee} has been discussed and calculated by Tom Verhoeff\cite{verhoeff2004optimal}, and an AI based on this work has been implemented by James Glenn\cite{glenn2006optimal}.
The strategy is to consider \emph{Yahtzee} a single player game where the goal is to maximize your score.  %Lidt mærkelig sætning
The state space for calculating this strategy is small enough that any modern computer will be able to perform the computations within a few hours.
We have described and implemented the single player AI algorithm using the work of James Glenn, which gives a in depth explanation of the implementation, theory, and testing of the solitaire \emph{Yahtzee} strategy (section \ref{sec:optimalSingleAI}).

James Glenn proposes an optimal solution for the multi-player game and mentions that the state space is of a size that makes it impossible to do - both in terms of space and time.
The aim of our project is not to calculate the optimal solution but instead to improve on the current state of the sub-optimal multi-player strategy  (section \ref{sec:subOptimalMPAI}). 
Our main goal is to show how it is possible with current technology, to implement a multi-player \emph{Yahtzee} strategy that has a consistently better performance in a multi-player setting than the optimal solitaire strategy, and can be computed and run in a reasonable amount of time (section \ref{sec:testsAndResults})).\footnote{i.e. at most \emph{days} of computation for preparation and \emph{seconds} of computation when playing.}

Claims have been made, indicating that such an algorithm has been already been crafted, but it has not been properly documented.
The work by Jakub Pawlewicz is, to our knowledge, the publication with the strongest documentation of how to design a well performing multi-player \emph{Yahtzee} strategy.
While the paper of Jakub Pawlewicz\citep{pawlewicz2011nearly} presents impressive results it does not test the solitaire strategy directly against a multi-player strategy.

We seek to show an algorithm for multi-player \emph{Yahtzee} that defeats the single player \emph{Yahtzee} AI more than $51\%$ of the time with at least $95\%$ confidence.
The strategy described in this article will only be viable for two-player \emph{Yahtzee}.

%We have implemented a \emph{Yahtzee} AI following a strategy somewhat similar to the optimal solitaire strategy, except that it switch between aggression based strategies depending on it's chance to beat the other players expected final score for a given state. 
%The idea is to base your choice of strategy from the \emph{standard deviation} in score which is known for each strategy that can be chosen in a current state. 
%The best strategy can then be chosen as the strategy with the highest \emph{cumulative normal distribution}, which describes the probability of beating score \emph{a} when having score \emph{b} and standard deviation $\sigma$.
%
%We will represent the theory behind the single player ai, how we have extended it into our competing multiplayer AI and show test results of how the two implementations performs when playing each other. 
%Finally we will test each AI against a restricted optimal \emph{Yahtzee} AI and calculate the average amount of mistake made by each of them.


%Introduction text
 %TODO Domæne (Yahtzee)
 %TODO SinglePlayer AI ref til Glenn
 %TODO Optimal MP AI (ikke vores impl) ref til Pawlewicz
 %TODO Idé (near optimal mp ai)!
