Since we can only have a limited amount of caches in memory we need to test which distribution of aggressivity levels we need to calculate caches for. We have arbitrarily used 21 caches\footnote{10 caches for negative aggressive strategies, 10 for positive aggressive strategies and 1 for the solitaire strategy(0 aggressivity)} for the final multiplayer AI. To decide upon the distribution of the caches we have tried several distribution strategies which have each been tested against the solitaire AI in a restricted game\footnote{All but 4 lower categories filled out by default}. 

Each test has been done using two \emph{assymetric boards}, that is two scoreboards with varying scores and categories set but each with 4 categories left. One board is therefore favored to win. We then use the principle of \emph{heads up duplicate matches} where first the first AI starts with the favorable board for n number of tests and then the boards are swapped so the first AI now have the unfavorable board for another n number of tests. The AI which scores the most wins in both two duplicate is then the superior. What we achieve from this form of testing is the possibility to quickly testing and calculation of the different distributions of the aggression levels, since it all calculation is done for only 4 categories left, while we still test the AI in a realistic end game setting, which is where solitaire AI performs the highest amount of suboptimal plays\citep{pawlewicz2011nearly}.

We have tested the following different distributions with the following results: %TODO show table of the different strategies performance against the solitaire AI%

The way we conduct our tets of the multiplayer and solitaire AI's performance is through monte carlo testing. We therefore have to take into consideration that the results are subject to variance and we therefore have to show that our results does prove that our multiplayer strategy beats the solitaire with a high amount of significance. We can calculate significance as: %TODO add formula about signficance%
As the formula shows the significance is highly reliant on the number of games played and the difference in win rate between the two strategies. After completing $100.000$ we got the following result: %TODO add small table showing win rate of both AI's and final significance%


We have used our analysis of the optimal multiplayer AI to create a restricted form of it. The reduction is to play the game with the upper categories already set. Thus the number of states becomes $14+9+1 = 24\mbox{bits}$, taking $2^{24-19} = 2^5 = 32$ times as long as the single player AI\footnote{In practice it took much less time than expected because of a high number of unreachable states}. The motivation for creating this AI is to test it respectively against single player AI and multiplayer AI. %TODO beskriv resultaterne af multiplayer og singleplayer AI mod optimal multiplayer AI%